\documentclass{article}

\usepackage{ctex}
\usepackage{xltxtra}
\usepackage{mflogo}
\usepackage{graphics}
\usepackage{array}
%\usepackage[thmmarks]{ntheorem} %定理环境
\usepackage{amsmath}

\bibliographystyle{plain}
\graphicspath{{D:/},{pics/}}
\newcommand\degree{^\circ}
\title{\songti 第一个文件}
\date{\today}
\author{\kaishu 张三}

\newcommand\abc{\text{666666}}

%正文区有且仅有一个document环境
\begin{document}
	\maketitle
	\tableofcontents
	
	\section{上下标}
	\subsection{a}
	这是一个公式$a_0$:$x^{100}+x+1$\\
	\subsection{b}
	这是一个公式$a_1$:\(x^{x+1}+x+1\)\\
	\subsection{c}
	这是一个公式$a_2$:\begin{math}
		x^2+x+1 \end{math}\\
	\subsection{d}
	$y = \arcsin^{-1} x$
	
	$y = \sqrt{x^{100}+x+1}$
	
	$y = \sqrt[4]{x^{100}+x+1}$
	\subsection{e}
	$3/4$ $\frac{3}{4}$
	\subsection{f}
	这是一个公式:
	$$\alpha^3 + \beta^3 + \gamma^3 = 0$$
	
	这是公式\ref{eq:eq1}:
	\begin{equation}
		\alpha^3 + \beta^3 + \gamma^3 = 0 \label{eq:eq1}
	\end{equation}

	这是公式\ref{eq:eq2}:
	\begin{equation*}
	\alpha^3 + \beta^3 + \gamma^3 = 0 \label{eq:eq2}
	\end{equation*}

	\begin{equation}
		min \quad f(x)=x_1+x_2
	\end{equation}
	$$s.t \quad g_1(x)=4-x_1^2-x_2^2 \ge 0$$
	
	\section{d2}
	$\alpha^3 + \beta^3 + \gamma^3 = 0$ 
	\section{矩阵}
	\subsection{a}
	\[
	\begin{vmatrix}		%Vv bB ...
		1 & 0 \\
		0 & 1
	\end{vmatrix} \qquad
	\]
	\subsection{b}
	\[
	\begin{Bmatrix}
		a_{11}^2 & \dots & a_{1n}^2\\
		         & \ddots & \vdots\\
		0        & & a_{nn}^2
	\end{Bmatrix}_{n \times n}	 \qquad
	\]
	\subsection{c}
	\[
	\begin{pmatrix}
	\begin{matrix}
		1&0 \\
		0&1
	\end{matrix}& \text{\Large 0} \\ \text{\Large 1} &
	\begin{matrix}
		1&0 \\
		0&1
	\end{matrix}
	\end{pmatrix}
	\]
	\subsection{d}
	\[
	\begin{bmatrix}
		a_11&a_12&\dots&a_1n\\
		&a_22&\dots&a_2n\\
		& & \ddots&\vdots\\
		 & & &a_nn	
	\end{bmatrix}
	\]
	\subsection{e}
	\[
	\begin{bmatrix}
		a_11&a_12&\dots&a_1n\\
		&a_22&\dots&a_2n\\
		& & \ddots&\vdots\\
		\multicolumn{2}{c}{\raisebox{1.4ex}[1pt]{\Huge 0} }& &a_nn	
	\end{bmatrix}
	\]
	\subsection{f}
	\[
	\begin{bmatrix}
		1 & \frac{1}{2} & \dots & \frac{1}{n}\\
		\hdotsfor{4}\\
		1 & \frac{2}{3} & \dots & \frac{2}{n}
	\end{bmatrix}
	\]
	\subsection{g}
这是一个矩阵:
	\begin{math}
		\left(
		\begin{smallmatrix}
			x&y\\z&b
		\end{smallmatrix}
		\right)
	\end{math}
来表示
	\subsection{h}
	\[
	\begin{array}{c|c}
		10000&2\\
		\hline
		2&\frac{1}{mssdn}
	\end{array}
	\]
	
	\section{多行公式}
	\subsection{a}
	\begin{gather} %带符号
		a+b=b+a\\
		ab=ba
	\end{gather}
	
	\begin{gather*}
	a+b=b+a\\
	ab=ba
	\end{gather*}

	\begin{gather}
		a+b=b+a \notag \\
		ab=ba \notag
	\end{gather}

	\subsection{b}
	
	\begin{align} % &放在哪,哪里就是正中间
	y &= x^3 + x^2 + 1\\
	y = x^3 + &x^2 + 1\\
	y = x^3 + x^2 + &1\\
	\end{align}

	\begin{equation}
		\begin{split}
			y &= x^3 + x^2 + 1\\
			&=x^3 + x^2 + 1
		\end{split}
	\end{equation}

	\begin{equation}
		D(x)=\begin{cases}
			1,&\text{if} \quad x \in \text{D};\\
			0,&\text{if} \quad x \in \text{B}.
		\end{cases}
	\end{equation}
	\section{参考文献引用}
	这是一篇参考文献\cite{Lu2018Single}
	
	\nocite{*}
	\bibliography{test}
	
	\section{d6}
	 
	\texttt{你好}\\
	\textsf{你好}\\
	%\includegraphics[scale=0.2]{wallhaven-6o2de6}
	
	{\rmfamily glhf}\\ \par{\ttfamily glhf}\\
	{\tiny This is our homework.}\\
	{\normalsize This is our homework.}\\
	\zihao{5} 你好\\
	\emph{\textit{\textbf{glhf}}}\\
	\textit{\textbf{glhf}}
	\quad %一个空格
	
	$f(x)=x^3+x^2+1$.
	$$f(x)=x^3+x^2+1$$
	$\ C=30\degree$
	
	\begin{equation}
		f(x)=x^2+x+1.
	\end{equation}

	Hello world!\\
	
	\LaTeX{}中的插图\ref{table1}:
	\begin{table}[t]
		\centering
		\caption{ 标题}	\label{table1}
		\begin{tabular}{|c| c| p{1cm}|}	
			\hline 
			姓名 & 性别 & 年龄\\
			\hline \hline
			tony & male & 200000000000\\
			\hline \hline
			miney & female & 21\\
			\hline 
		\end{tabular}
	\end{table}

	\section{last}
	\abc
	
\end{document}